\documentclass[12pt]{article}
\usepackage[top=1.25in, bottom=1.25in, left=1in, right=1in]{geometry}
%\usepackage[top=0.9in, bottom=0.9in, left=0.5in, right=0.5in]{geometry}
\usepackage{amsmath}
\usepackage{amsfonts}
\usepackage{enumitem}

%\usepackage{titlesec}

% Perhaps not semantically ideal, but it works.
%\titleformat*{\section}{\large\bfseries}

\setenumerate{parsep=0em, listparindent=\parindent}

\newcommand{\E}{\mathrm{E}}
\newcommand{\Var}{\mathrm{Var}}
\newcommand{\Cov}{\mathrm{Cov}}
\newcommand{\Prob}{\mathrm{P}}

\title{Homework 1}
\author{Benjamin Noland}
\date{}

\begin{document}

\maketitle

\section*{Exercise 1.1}

Fix an initial state $x \in \mathbb{Z}_n$. We want to transition to an arbitrary state $y \in \mathbb{Z}_n$. Note that there are two simple paths in the state diagram leading from $x$ to $y$. If one of these simple paths follows $k$ edges, then the other follows $(n - k)$ edges. Thus if $k$ is odd, then $(n - k)$ is even since $n$ is odd. Thus there exists an even-lengthed simple path from $x$ to $y$. This means it is possible to transition from $x$ to $y$ in an even number of steps, say $m$. Note that $(n - 1) - m$ is even. Thus after following $m$ steps to get from $x$ to $y$, we can make $(n - 1) - m$ transitions between $y$ and one of its neighboring states, returning at the end to state $y$. Thus we can transition from $x$ to $y$ in $(n - 1)$ steps, so that $P^{n-1}(x, y) > 0$.

Next I claim that if $0 \leq s < n - 1$, then there exists a state $y \in \mathbb{Z}_n$ with $P^s(x, y) = 0$. Suppose that it is possible to transition from $x$ to any state $y$ in $s$ steps. This implies that $s$ must be even (this includes the possibility of $s = 0$), since otherwise it would be impossible to transition from $x$ to $x$ itself. Now suppose $y$ is a state neighboring $x$ in the state diagram (i.e., $x$ and $y$ are separated by a single edge). Since $s$ is even, and we are trying to transition from $x$ to $y$ in $s$ steps, we cannot transition along this single edge to get from $x$ to $y$. Thus we must transition through the remaining $(n - 1)$ edges in order to get from $x$ to $y$. However, since $s < n - 1$, this means it is impossible to transition from $x$ to $y$ in $s$ steps. In particular, $P^s(x, y) = 0$.

Therefore the minimum value of $t \geq 0$ such that $P^t(x, y) > 0$ for every pair of states $x$ and $y$ is $t = n - 1$.

\section*{Exercise 1.4}

I will assume that the definition of \textbf{\emph{tree}} taken here assumes that a tree contains at least two vertices, since otherwise we could consider a graph containing a single vertex to be a tree, yet this tree has no leaves.

\begin{enumerate}[label=(\alph*)]
\item
I will proceed by induction on the number $n$ of vertices in the tree. Suppose $n = 2$. In this case, since any tree is connected and acyclic, each vertex has a single incident edge, and hence degree 1. Thus both vertices are leaves. Now let $n > 2$, and suppose that every tree with $n$ vertices has a leaf. Suppose that $T$ is a tree with $(n + 1)$ vertices, and let $T'$ be a subtree of $T$ with $n$ vertices, i.e., $T'$ results from deleting a single vertex $v$ from $T$. Since $T$ is connected, $\deg(v) \geq 1$. We must also have $\deg(v) \leq 1$. To see this, suppose that $\deg(v) > 1$, and let $v_1$ and $v_2$ be two neighbors of $v$. Then there exists a simple path $v_1 \to v \to v_2$ in $T$, and since trees are acyclic, every path from $v_1$ to $v_2$ must pass through $v$. However, since $T'$ consists precisely of $T$ with $v$ removed, there is no path from $v_1$ to $v_2$ in $T'$, and hence $T'$ is disconnected, a contradiction. Therefore $\deg(v) = 1$, so that $v$ is a leaf in $T'$. Hence every tree contains a leaf by way of induction.

\item
I will proceed by induction on the number $n$ of vertices in the tree. Suppose $n = 2$, and let $v_1$ and $v_2$ be the tree's sole vertices. Then since any tree is connected and acyclic, there exists a unique simple path $v_1 \to v_2$ along the single edge connecting $v_1$ and $v_2$. Now let $n > 2$, and suppose the result holds for any tree with $n$ vertices. Let $T$ be a tree with $(n + 1)$ vertices, and let $T'$ be a subtree of $T$ with $n$ vertices, i.e., $T'$ results from deleting a single vertex $v$ from $T$. By the argument in part (a), $v$ must be a leaf in $T$, and thus has a sole neighboring vertex $w$ (note that $w$ is a vertex in $T'$ as well). By assumption, there exists a unique simple path from any vertex in $T'$ to $w$, and hence to $v$ via the single edge connecting $w$ to $v$. Hence there exists a unique simple path between any two vertices in $T$, and so the result follows by way of induction.

\item
I will proceed by induction on the number $n$ of vertices in the tree. When $n = 2$, there is a single edge connecting the two vertices, and so both vertices are leaves. Now let $n > 2$, and suppose that any tree with $n$ vertices contains at least two leaves. Let $T$ be a tree with $(n + 1)$ vertices, and let $T'$ be a subtree of $T$ with $n$ vertices, i.e., $T'$ results from deleting a single vertex $v$ from $T$. By the argument in part (a), $v$ must be a leaf, and thus has a sole neighboring vertex $w$ (note that $w$ is a vertex in $T'$ as well). By assumption, $T'$ has at least two leaves, and so $T'$ has a leaf $z$ with $z \neq w$. Since $z$ is a vertex in $T$ as well, and $\deg(z)$ is unaffected by the removal of $v$ (since $v$ and $z$ are not neighbors), $z$ is a leaf in $T$. Thus $v$ and $z$ are two leaves in $T$, so that $T$ contains at least two leaves. Therefore every tree contains at least two leaves by way of induction.

\end{enumerate}

\section*{Exercise 1.5}

I will proceed by (strong) induction on the number $n$ of vertices in the tree $T$. For the sake of simplicity, in what follows I will refer to an acyclic graph containing a single vertex as a ``tree''. Let $n \geq 2$, and suppose that for every tree $T$ with $1 \leq m < n$ vertices it is possible to transform a proper 3-coloring $\mathcal{C}_1(T)$ of $T$ into any other proper 3-coloring $\mathcal{C}_2(T)$ of $T$ by changing the color of a single vertex at a time. Now let $T$ be a tree with $n$ vertices, and delete a leaf $v$ of $T$ to yield a subtree $T'$ of $T$ with $(n - 1)$ vertices ($T$ has a leaf by Exercise~1.4~(a)). Let $w$ denote the sole neighbor of $v$ in the original tree $T$. Consider proper 3-colorings $\mathcal{C}_1(T)$ and $\mathcal{C}_2(T)$ of $T$. Since $T'$ is simply $T$ with a single leaf removed, $\mathcal{C}_1(T)$ and $\mathcal{C}_2(T)$ induce well-defined proper 3-colorings $\mathcal{C}_1(T')$ and $\mathcal{C}_2(T')$, respectively, on $T'$. By assumption it is possible to transform $\mathcal{C}_1(T')$ into $\mathcal{C}_2(T')$ by changing the color of a single vertex at a time. Thus we can, step by step, recolor $T'$, while at each step changing the color of $v$ so that $v$ and $w$ have different colors (this yields a valid proper 3-coloring at each step, since $w$ is the sole neighbor of $v$). We therefore eventually transform $\mathcal{C}_1(T')$ into $\mathcal{C}_2(T')$. At this stage we simply change the color of $v$ in order to obtain $\mathcal{C}_2(T)$. Thus it is possible to transform $\mathcal{C}_1(T)$ into $\mathcal{C}_2(T)$ as well. It therefore follows, by way of induction, that the graph whose vertices are proper 3-colorings of a given tree is connected.

\section*{Additional problem}

Throughout, $\mathbb{Z}^+$ denotes the non-negative integers.

\begin{enumerate}
\item
The initial distribution is given by $\mu_0 = \delta_1$. That is, for any state $x \in \mathbb{Z}^+$,
\begin{equation*}
\mu_0(x) = \begin{cases}
1 &\quad \text{if $x = 1$} \\
0 &\quad \text{if $x \neq 1$}
\end{cases}.
\end{equation*}

\item
Define the following independent random variables:
\begin{align*}
Z &\sim \mathrm{discrete\ unif}\{-5, -4, \ldots, 5\} \\
U & \sim \mathrm{unif} [0, 1]
\end{align*}
and form the random vector $(Z, U)$. Let $\Lambda = \{-5, -4, \ldots, 5\} \times [0, 1]$. Define a map ${f : \mathbb{Z}^+ \times \Lambda \to \mathbb{Z}^+}$ by
\begin{equation*}
f(x, z, u) = \begin{cases}
x + z &\quad \text{if $u \leq \alpha(x, x+z)$} \\
x &\quad \text{if $u > \alpha(x, x+z)$}
\end{cases}.
\end{equation*}
The map $\alpha : \mathbb{Z}^+ \times \mathbb{Z}^+ \to \mathbb{R}$ denotes the acceptance probability, defined by
\begin{equation*}
\alpha(x, y) = \min\left\{1, \frac{\pi(y)}{\pi(x)} \right\},
\end{equation*}
where $\pi$ is the pmf for a $\mathrm{binomial(100, 0.75)}$ random variable. The random vector $(Z, U)$, paired with the map $f$, defines a random mapping representation for the chain in question.

\item
The transition probabilities can be obtained from the random mapping representation given in part~(2). In general, for any states $x, y \in \mathbb{Z}^+$,
\begin{equation} \label{eq:1}
\begin{split}
P(x, y) &= \Prob \{f(x, Z, U) = y\} \\
&= \Prob \{x + Z = y, U \leq \alpha(x, x+Z)\} + \Prob \{x = y, U > \alpha(x, x+Z)\} \\
&= \Prob \{Z = y - x, U \leq \alpha(x, y)\} + \Prob \{x = y, U > \alpha(x, x+Z)\} \\
&= \Prob \{Z = y - x\} \Prob \{U \leq \alpha(x, y)\} + \Prob \{x = y\} \Prob \{U > \alpha(x, x+Z)\}.
\end{split}
\end{equation}
There are several cases to consider. Let $\mathcal{T}_x = \{x-5, x-4, \ldots, x+5\}$. If $y \not \in \mathcal{T}_x$, then $x \neq y$ and ${\Prob \{Z = y - x\} = 0}$, so that $P(x, y) = 0$ as well by (\ref{eq:1}). Now suppose $y \in \mathcal{T}_x$. If $y \neq x$, then
\begin{equation*}
P(x, y) = \Prob \{Z = y - x\} \Prob \{U \leq \alpha(x, y)\} = \frac{1}{11} \alpha(x, y).
\end{equation*}
On the other hand (noting that $\alpha(x, x) = 1$), we have
\begin{align*}
P(x, x) &= \Prob \{Z = 0\} \Prob \{U \leq \alpha(x, x)\} + \Prob \{U > \alpha(x, x+Z)\} \\
&= \frac{1}{11} \alpha(x, x) + \sum_{z=-5}^5 \Prob \{U > \alpha(x, x+Z) | Z = z\} \Prob \{Z = z\} \\
&= \frac{1}{11} + \frac{1}{11} \sum_{z=-5}^5 \Prob \{U > \alpha(x, x+z)\} \\
&= \frac{1}{11} + \frac{1}{11} \sum_{z=-5}^5 (1 - \alpha(x, x + z)).
\end{align*}

\item
Consider states $x, y \in \mathbb{Z}^+$. If $y = x$, then this relation holds trivially, so assume $y \neq x$. Let
\begin{equation*}
\mathcal{T}_x = \{x-5, x-4, \ldots, x+5\} \quad \text{and} \quad \mathcal{T}_y = \{y-5, y-4, \ldots, y+5\}.
\end{equation*}
If $y \not \in \mathcal{T}_x$, then $x \not \in \mathcal{T}_y$, so that $P(x, y) = P(y, x) = 0$ by part~(3), and the relation holds trivially. On the other hand, if $y \in \mathcal{T}_x$, then $x \in \mathcal{T}_y$, and by part~(3),
\begin{equation*}
P(x, y) = \frac{1}{11} \alpha(x, y) = \frac{1}{11} \min\left\{1, \frac{\pi(y)}{\pi(x)} \right\}.
\end{equation*}
(Since $x \in \mathcal{T}_y$, an analogous expression holds for $P(y, x)$). Assume without loss of generality that $\pi(y) / \pi(x) \geq 1$. Then $\pi(x) / \pi(y) \leq 1$, and we have
\begin{equation*}
P(x, y) = \frac{1}{11} \quad \text{and} \quad P(y, x) = \frac{1}{11} \frac{\pi(x)}{\pi(y)}.
\end{equation*}
Hence,
\begin{equation*}
\pi(x) P(x, y) = \frac{1}{11} \pi(x) = \pi(y) \left(\frac{1}{11} \frac{\pi(x)}{\pi(y)}\right) = \pi(y) P(x, y).
\end{equation*}
Therefore,
\begin{equation*}
\sum_{x \in \mathbb{Z}^+} \pi(x) P(x, y) = \sum_{x \in \mathbb{Z}^+} \pi(y) P(y, x) = \pi(y) \sum_{x \in \mathbb{Z}^+} P(y, x) = \pi(y),
\end{equation*}
so that the distribution $\pi \sim \mathrm{binomial(100, 0.75)}$ is stationary for this chain.

\item
The distribution described by the histogram is approximately the stationary distribution $\pi$ described in part~(4). In particular, the histogram has the correct shape and is approximately centered at $\E(\pi) = 75$.

\end{enumerate}

\end{document}
